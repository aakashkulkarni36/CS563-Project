\documentclass[12pt, letterpaper]{report}
\usepackage{setspace} 
\usepackage{geometry}
\usepackage{marginnote}
\usepackage{xcolor}
\usepackage{ulem} 
\usepackage{amssymb}
\usepackage{graphicx}
\usepackage{listings}

\definecolor{questionColor}{RGB}{41,128,185}

\setlength{\parindent}{0pt}

\setstretch{1.4}

\geometry{
    left=0.75in,
    right=0.75in,
    top=1.1in,
    bottom=1.1in,
}

\setlength{\parskip}{1em}

% \newcommand{\qa}[2]{
%   \textcolor{questionColor}{\textbf{\textit{#1}}} \\
%   #2 
% }


\title{\textbf{ Project Idea Proposal:\\ Analysis of Fault Localization Techniques on REST APIs
\\ CS563 --- Spring 2024}}

\author{Aakash Kulkarni, Soon Song Cheok
\\ kulkaraa@oregonstate.edu, cheoks@oregonstate.edu
}

\begin{document}
\maketitle


\end{document}



% For each API A:
%     For each Fault F in A:
%         For each FaultLocalization T that supports A: # T is Technique
%             Try:
%                 Result = localize(Fault, using T)
%                 o/p = compare(result, groundTruth)

%                 if o/p == true:
%                     localize(Fault)
%                 else:
%                     unlocalize(Fault)



% Overall Result:
% Number of Faults localized and number of faults not localized.
% Now, what of kind of faults can be localized and at what precision, and what number of faults are not localized, what kind of faults are no localized?
